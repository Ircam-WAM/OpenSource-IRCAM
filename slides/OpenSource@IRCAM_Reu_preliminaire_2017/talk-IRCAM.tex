\documentclass[xcolor=dvipsnames]{beamer}

\usepackage[utf8]{inputenc}
\usepackage[french]{babel}
\usepackage{url}
\usepackage{lmodern}
\usepackage{listings}
\usepackage{graphicx}
\usepackage{xcolor}
\usepackage{textcomp}
\usepackage{hyperref}
\usepackage[T1]{fontenc}
\usepackage{xcolor}
\usepackage{syntax}
\usepackage{color}
\usepackage{subcaption}
\usepackage{tikz}

\pagenumbering{gobble}
\pagestyle{empty}

\definecolor{light-gray}{gray}{0.60}

\setbeamertemplate{navigation symbols}{}
% \addtobeamertemplate{footline}{}
% \setbeamertemplate{footline}{\hspace*{.5cm}\scriptsize{\insertauthor\hspace*{50pt} \hfill\insertframenumber\hspace*{.5cm}}}

\AtBeginSection[]
{
  \begin{frame}<beamer>{Menu}
    \tableofcontents[currentsection,hideothersubsections]
  \end{frame}
}

% \AtBeginSubsection[]
% {
%   \begin{frame}<beamer>{Le menu}
%     \tableofcontents[currentsection,currentsubsection,hideothersubsections,subsectionstyle=show/shaded/hide]
%   \end{frame}
% }


\setbeamertemplate{section in toc}{%
  \textcolor{MidnightBlue}{$\blacktriangleright$ \inserttocsection}
}

\setbeamertemplate{subsection in toc}{%
\leavevmode\leftskip=5.65ex%
  \llap{\raisebox{0.2ex}{\textcolor{MidnightBlue}{$\blacktriangleright$}}\kern1ex}%
  \textcolor{MidnightBlue}{\inserttocsubsection}\par%
}

\usecolortheme{seahorse}
\usecolortheme{rose}
\useoutertheme{infolines}

\usecolortheme[named=SkyBlue]{structure}
\setbeamercolor{block title}{fg=MidnightBlue}

\title[Open-source]{Open-source à l'IRCAM}
\subtitle{Réunion préliminaire}
\author[Pierre Talbot]{\textbf{Pierre Talbot}\\
\texttt{(talbot@ircam.fr)}}
\institute[IRCAM]{Institut de Recherche et Coordination Acoustique/Musique (IRCAM)}
\date[]{17 mai 2017}

\begin{document}
\maketitle

\section{Introduction}

\begin{frame}
\frametitle{Introduction}
\begin{itemize}
\item Faire un tour d'horizon de l'open-source et son applicabilité dans le cadre de l'IRCAM.
\item Développer des problématiques et proposer des solutions.
\end{itemize}
\begin{block}{Sources}
Présentation en partie basée sur ``Les livrets bleus du Logiciel Libre'': ``Modèles économiques'' et ``Fondamentaux juridiques'' de l'IRILL (slides marquées d'une astérisque).
\end{block}
\end{frame}

\begin{frame}
\frametitle{Le logiciel libre*}

\begin{block}{Qu'est-ce donc ?}
``Le logiciel libre est caractérisé par sa liberté d'utilisation, un développement collaboratif et sa redistribution à une communauté de contributeurs et d'utilisateurs.''
\end{block}

\begin{block}{Pourquoi l'open-source}
\begin{itemize}
\item Pour la recherche ;
\item Pour construire une communauté ;
\item Pour la valorisation d'un produit.
\end{itemize}
\end{block}

\end{frame}

\begin{frame}
\frametitle{Pour la recherche}
\begin{itemize}
\item Réplicabilité des résultats scientifiques. Exemple : l'implémentation d'un algorithme décrit dans un article. (RR_Webinar, ReScience, activepapers.org)
\item Visibilité du projet, possibilité d'utilisation et collaboration avec d'autres scientifiques.
\item Citer une version exacte d'un logiciel dans un papier (rigeur scientifique).
\end{itemize}
\end{frame}

\begin{frame}
\frametitle{Pour construire une communauté}
\begin{itemize}
\item S'imposer ``implicitement'' comme la référence grâce à la communauté autours du logiciel.
\item Contributeurs externes potentiellement ``recrutable'' (développeurs, stage master, doctorants, docteur étendant des logiciels,...)
\item Attraction de développeurs et chercheurs de qualité (notamment programmes open-source associées : cf. Google Summer of Code, ...)
\end{itemize}
\end{frame}

\begin{frame}
\frametitle{Pour la valorisation d'un produit*}
\begin{itemize}
\item Visibilité accrue auprès de clients potentiels.
\item Marketing plus simple puisque la solution est facilement testable.
\item Formation facilitée : au niveau universitaire, ... (si les logiciels sont gratuits / ou un fragment).
\end{itemize}
\end{frame}

\begin{frame}
\frametitle{Open-source à l'IRCAM}
Création d'une fondation open-source pour :
\begin{itemize}
\item Consolider l'IRCAM dans sa position de leader dans les logiciels de son ;
\item Profiter d'un développement plus efficace ;
\item Agrandir la communauté d'utilisateurs ET de contributeurs ;
\item Tout en gardant le contrôle sur les aspects économiques.
\end{itemize}
\end{frame}

\section{Modèles économiques}

\begin{frame}[fragile]
\frametitle{Les acteurs*}

\begin{itemize}
\item Fondation : Son rôle est de mutualiser les efforts de R\&D afin de répondre à des besoins génériques.
\item Éditeur : Industrialiser une solution open-source en l'adaptant aux besoins des entreprises.
\item Intégrateur : Intègre dans une entreprise le logiciel et fournit la maintenance.
\end{itemize}

\begin{example}
Fondation Linux $\rightarrow$ Entreprise RedHat (Éditeur/Intégrateur) $\rightarrow$ Intel/Microsoft/... (Intégrateur/Client)
\end{example}

\end{frame}

\begin{frame}
\frametitle{Problématique 1 : IRCAM en tant qu'acteur}

\begin{itemize}
\item Fondation : Création de logiciels issus d'effort R\&D : (OpenMusic, Orchids, \ldots).
\item Éditeur : Le forum distribue et vend des solutions complètes.
\item Intégrateur : Les formations aux logiciels (événements du forum).
\end{itemize}

\begin{block}{Objectif}
Délimiter les rôles de chaque entité et assurer leur bonne collaboration.
\end{block}
\end{frame}

\begin{frame}
\frametitle{Modèles économiques*}

\begin{itemize}
\item Doubles licences :
  \begin{enumerate}
  \item Logiciel open-source disponible sous licence à copyleft fort (GPL, AGPL) : pas de possibilité d'intégration.
  \item Licence propriétaire payante pour les entreprises.
  \end{enumerate}
\item \textit{Open core} : Développement d'un noyau technologique open-source et d'un produit payant au dessus de ce noyau.
\item \textit{Cloud} : Logiciel open-source et mise à disposition payante sous forme de service web.
\item Open-source professionnel : Logiciel open-source. Profits tirés de ``la maintenance et du support associés au logiciel édité'' (Wikipedia).
\end{itemize}
\end{frame}

\begin{frame}
\frametitle{Problématique 2 : Quel modèle pour l'IRCAM}

\begin{itemize}
\item Diversité des contenus : code informatique (application ``end-user'' ou librairies), patch, création musicales, base de données de son, \ldots.
\item IRCAM en tant qu'institut de recherche et éditeur de logiciel : objectifs pas toujours les mêmes.
\item Comment commencer ? Encourager les nouveaux projets à s'orienter vers une solution open-source et fournir un support d'aide ?
\item Attention aux dérives (notamment cas Antescofo).
\end{itemize}
\end{frame}

\section{Aspects juridiques}

\begin{frame}
\frametitle{Licences open-source*}

\begin{itemize}
\item Ces licences ont été écrite pour pouvoir être utilisée dans un cadre juridique si besoin.
\item Énormément de licences mais finalement peu de catégories :
  \begin{enumerate}
    \item \textit{Copyleft fort} : Obligation de diffuser les œuvres dérivées sous la même licence, effet contaminant : GPL.
    \item \textit{Copyleft faible} : Permet l'utilisation du code dans un cadre potentiellement non libre tant que les modifications sur le projet en lui-même sont rediffusées : LGPL, Mozilla Public Licence.
    \item \textit{Non-copyleft} : Licences permissives : MIT, BSD, Apache.
  \end{enumerate}
\end{itemize}

\end{frame}

\section{En pratique}


\begin{frame}
\frametitle{Niveau d'utilisation de l'open-source}

\begin{enumerate}
\item Uni-directionnel : distribution du code source sous forme d'archive, contact par mail, peu d'interaction.
\item Bi-directionnel :
  \begin{itemize}
  \item Mise à disposition des sources sur une plateforme (Github).
  \item Utilisation du \textit{bug tracker} et retour des utilisateurs.
  \item Review des pull requests des contributeurs.
  \end{itemize}
\item Communautaire : Voir projet Rust (organisation open-source au cœur du mode de fonctionnement)
  \begin{itemize}
    \item Communication intensive avec la communauté.
    \item Mentorat pour les nouveaux contributeurs et utilisateurs (forum, IRC, ...).
    \item Roadmap publique et collaborative.
    \item Organisation hiérarchique en groupe de travail, rapport d'activité publique.
    \item Nouvelles fonctionnalités suivent un système de RFCs.
    \item \ldots
  \end{itemize}
\end{enumerate}
\end{frame}

\begin{frame}
\frametitle{Mesures concrètes}

Quelques mesures concrètes applicables rapidement (pour commencer) :

\begin{itemize}
\item Création d'un manifeste contenant les licences recommandées par l'IRCAM et information relative à l'open-source.
\item Création d'un compte "Institution" sur GitHub pour l’agrégation des softwares de l'IRCAM.
\item Workshop servant de moteur de diffusion de ces nouvelles idées et mouvance à l'IRCAM.
\item Conseil auprès de l'IRILL.
\end{itemize}

\begin{block}{Github}
\begin{itemize}
  \item + de 12 millions d'inscrits.
  \item Projets open-source majeurs sur cette plateforme.
  \item Vite dit : Beaucoup mieux que notre forge IRCAM.
\end{itemize}
\end{block}

\end{frame}

\begin{frame}
\frametitle{Un workshop ?}
Quelques idées d'orateurs :
\begin{itemize}
\item Felix S, fondation Mozilla, projet Rust : Éco-système autour de Rust extrêment développé et bien pensé.
\item Roberto Di Cosmo, directeur de l'IRILL, fondateur de Software Heritage : Open-source et industriels. (fallback : Emmanuel Chailloux, IRILL).
\item Guillaume, IRCAM : expérience industriel.
\item \ldots
\end{itemize}
\end{frame}

\begin{frame}
\frametitle{Conclusion}

Adopter l'open-source, c'est prendre le wagon en marche et s'assurer de la pérennité des résultats de recherche ainsi que donner de nouvelles perspectives économiques pour l'IRCAM.

\end{frame}

\begin{frame}
\frametitle{Merci de votre attention.}
\begin{center}
% \includegraphics[scale=0.5]{../../../images/question.jpg}
\end{center}
\end{frame}


\end{document}